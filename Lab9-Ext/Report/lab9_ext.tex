\documentclass{cmc}
\usepackage{makecell}
\begin{document}

\pagestyle{fancy}
\lhead{\textit{\textbf{Computational Motor Control, Spring 2019} \\
    Salamander exercise, Lab 9 Extension, GRADED}} \rhead{Student \\
  Names}

\section*{Student names: \ldots (please update)}

\textit{Instructions: Update this file (or recreate a similar one,
  e.g.\ in Word) to prepare your answers to the questions. Feel free
  to add text, equations and figures as needed. Hand-written notes,
  e.g.\ for the development of equations, can also be included e.g.\
  as pictures (from your cell phone or from a scanner).
  \textbf{\corr{This lab is graded.}} and needs to be submitted before
  the \textbf{\corr{Deadline : 07-06-2019 Midnight. You only need to
      submit one final report for all of the following exercises
      combined henceforth.}} Please submit both the source file
  (*.doc/*.tex) and a pdf of your document, zipped file called
  \corr{final\_report\_extension\_name1\_name2\_name3.zip} where
  name\# are the team member’s last names.  \corr{Please submit only
    one report per team!}}
\\

\section*{Propose a potential additional study that could be performed
  in simulation and with the real salamander.  This should be written
  like a research proposal using the questions listed below and should not exceed
  2 pages (including figures and references). You are free to choose
  any topic related to sensorimotor coordination and locomotion of the
  salamander.}
\corr{NOTE : The proposal should be just text (possibly with some
  figures), there is no need to perform the actual numerical
  experiments!}
\label{sec:research-proposal}

\subsection{Provide a scientific question}
\subsection{Formulate a hypothesis corresponding to the scientific question}
\subsection{Describe an experiment in simulation that could be
  performed to test the hypothesis}
\subsection{Specify which type of simulation (e.g. neural circuits +
  biomechanics, neural circuits alone, etc.), which level of
  abstraction, and which assumptions (cf the modeling steps presented
  in the course)}
\subsection{Specify a corresponding experiment that could be performed
  with the real animal}
\subsection{Discuss what you expect and what could be learned from
  those experiments (in simulation and real)}
\subsection{Refer to and include a short bibliography with relevant
  literature (Example \cite{ijspeert2007swimming})}


\newpage
\bibliography{lab9_ext}
\label{sec:references}
\bibliographystyle{ieeetr}


% \newpage

% \section*{APPENDIX}
% \label{sec:appendix}

\end{document}

%%% Local Variables:
%%% mode: latex
%%% TeX-master: t
%%% End: